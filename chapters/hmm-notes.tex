\section{HMM相关知识点总结}
\subsection{basic infomation}
设Q是所有可能状态的集合,V是所有可能观测的集合。其中$N$是可能的状态数,$M$是可能的观测数。
\begin{center}
$Q=\{q_1, q_2, ..., q_N\}$, $V = \{v_1, v_2, ..., v_M\}$
\end{center}

$I$是长度为$T$的状态序列,$O$是对应的观测序列。
\begin{center}
$I=\{i_1, i_2, ..., i_T\}$, $O = \{o_1, o_2, ..., o_T\}$
\end{center}

$A$为状态转移矩阵,如公式\ref{trans-matrix},其中$a_{ij}=P(i_{t+1}=q_j|i_t=q_i)$,$i=1,2,...,N; j=1,2,...,N$,是在时刻$t$处于状态$q_i$的条件下在时刻$t+1$转移到状态$q_j$的概率。
\begin{align}
\label{trans-matrix}
A=[a_{ij}]_{N\times{N}}
\end{align}

$B$是观测概率矩阵,如公式\ref{emit-matrix},其中$b_j(k)=P(o_t=v_k|i_t=q_j)$,$k=1,2,...,M; j=1,2,...,N$是$t$时刻处于状态$q_j$的条件下生成观测$v_k$的概率。
\begin{align}
\label{emit-matrix}
B=[b_{j}(k)]_{N\times{M}}
\end{align}

$\pi$是初始状态概率向量,如公式\ref{init-vector},其中$\pi_i=P(i_1=q_i)$,$i=1,2,...,N$。
\begin{align}
\label{init-vector}
\pi = (\pi_i)
\end{align}

HMM有三个基本问题:

(1)概率计算问题。给定模型$\lambda=(A,B,\pi)$和观测序列$O=(o_1, o_2, ..., o_T)$,计算在模型$\lambda$的条件下观测序列$O$出现的概率$P(O|\lambda)$。

(2)学习问题。已知观测序列$O=(o_1, o_2, ..., o_T)$,估计模型$\lambda=(A,B,\pi)$参数,使得在该模型下观测序列$P(O|\lambda)$最大,即用最大似然估计的方法估计参数。

(3)预测问题,也称为解码问题。已知模型$\lambda=(A,B,\pi)$和观测序列$O=(o_1, o_2, ..., o_T)$,求对给定观测序列条件概率$P(I|O)$最大的状态序列$I=\{i_1, i_2, ..., i_T\}$,即给定观测序列,求最有可能的对应的状态序列。

\subsection{概率计算问题}
\subsubsection{前向算法}

\subsubsection{后向算法}

\subsection{学习问题}

\subsection{解码问题}