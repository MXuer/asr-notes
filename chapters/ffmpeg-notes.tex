\chapter{FFmpeg和sox}
\section{FFmpeg}
\subsection{安装FFmpeg}
\href{https://www.jianshu.com/p/2b98e0f87720}{在Centos中利用yum安装FFmpeg}步骤如下:
\begin{enumerate}
\item 升级系统
\begin{lstlisting}[language = shell, numbers=left, 
         numberstyle=\tiny,keywordstyle=\color{blue!70},
         commentstyle=\color{red!50!green!50!blue!50},frame=shadowbox,
         rulesepcolor=\color{red!20!green!20!blue!20},basicstyle=\ttfamily]
sudo yum install epel-release -y
sudo yum update -y
sudo shutdown -r now
\end{lstlisting}
\item 安装Nux Dextop Yum源 \\
	由于CentOS没有官方FFmpeg rpm软件包。但是,我们可以使用第三方YUM源(Nux Dextop)完成此工作。
	\begin{itemize}
		\item CentOs 7
\begin{lstlisting}[language = shell, numbers=left, 
         numberstyle=\tiny,keywordstyle=\color{blue!70},
         commentstyle=\color{red!50!green!50!blue!50},frame=shadowbox,
         rulesepcolor=\color{red!20!green!20!blue!20},basicstyle=\ttfamily]
sudo rpm --import http://li.nux.ro/download/nux/RPM-GPG-KEY-nux.ro
sudo rpm -Uvh http://li.nux.ro/download/nux/dextop/el7/x86_64/nux-dextop-release-0-5.el7.nux.noarch.rpm
\end{lstlisting}
		\item CentOs 6
\begin{lstlisting}[language = shell, numbers=left, 
         numberstyle=\tiny,keywordstyle=\color{blue!70},
         commentstyle=\color{red!50!green!50!blue!50},frame=shadowbox,
         rulesepcolor=\color{red!20!green!20!blue!20},basicstyle=\ttfamily]
sudo rpm --import http://li.nux.ro/download/nux/RPM-GPG-KEY-nux.ro
sudo rpm -Uvh http://li.nux.ro/download/nux/dextop/el6/x86_64/nux-dextop-release-0-2.el6.nux.noarch.rpm
\end{lstlisting}
	\end{itemize}
\item 安装FFmpeg 和 FFmpeg开发包
\begin{lstlisting}[language = shell, numbers=left, 
         numberstyle=\tiny,keywordstyle=\color{blue!70},
         commentstyle=\color{red!50!green!50!blue!50},frame=shadowbox,
         rulesepcolor=\color{red!20!green!20!blue!20},basicstyle=\ttfamily]
sudo yum install ffmpeg ffmpeg-devel -y
\end{lstlisting}
\item 测试是否安装成功
\begin{lstlisting}[language = shell, numbers=left, 
         numberstyle=\tiny,keywordstyle=\color{blue!70},
         commentstyle=\color{red!50!green!50!blue!50},frame=shadowbox,
         rulesepcolor=\color{red!20!green!20!blue!20},basicstyle=\ttfamily]
ffmpeg
\end{lstlisting}
\end{enumerate}

备注:

(1)查看机器的centos版本:cat /etc/redhat-release

(2)在执行安装FFmpeg和FFmpeg包的时候,可能会出现一些错误,因为有些依赖包没有安装,看下未安装的包有哪些,然后去\href{https://pkgs.org}{pkgs}官网上去搜索下载对应的版本即可。

(3)\href{http://ffmpeg.org/}{FFmpeg}的使用参考资料
	\begin{itemize}
		\item \href{https://www.cnblogs.com/reach296/p/4002020.html}{reach296}的博客;
		\item \href{https://github.com/feixiao/ffmpeg}{feixiao的GitHub库};
		\item \href{https://juejin.im/post/5a59993cf265da3e4f0a1e4b}{ffprobe,ffplay ffmpeg常用的命令行命令}
	\end{itemize}

\section{sox}
sox的参考资料:
\begin{itemize}
	\item \href{https://rollingstarky.github.io/2018/12/18/processing-audio-with-sox/}{SoX — 音频处理工具里的瑞士军刀}
\end{itemize}

