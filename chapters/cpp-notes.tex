\chapter{C++学习笔记}
\section{C语言碎记}
\begin{enumerate}
\item system("pause")可以让程序输出的cmd窗口卡住,方便看程序输出结果,在VS中没啥用,VC中有点用,其来自于stdlib.h;

\end{enumerate}

\section{C++库}
\subsection{标准库}
{\bf <cassert>}:

此头文件原作为 <assert.h> 存在于 C 标准库。此头文件是错误处理库的一部分

{\bf <utility>}:

\section{C++中的预定义宏,\_\_FILE\_\_等}

\section{\#pragma once和\#ifndef作用与区别}


\section{C++指针} % (fold)
\label{sec:c_指针}
\begin{enumerate}
	\item 指针和字符串:先看下程序。
	\begin{lstlisting}[language=C++, numbers=left, 
         numberstyle=\tiny,keywordstyle=\color{blue!70},
         commentstyle=\color{red!50!green!50!blue!50},frame=shadowbox,
         rulesepcolor=\color{red!20!green!20!blue!20},basicstyle=\ttfamily]
		#include <iostream>
		#include <cstring>
		int main()
		{
			using namespace std;
			int x[3] = { 1,2,3 };
			int* p1 = x;
			char animal[20] = "bear";
			const char* bird = "wren";
			char* p = animal;
			cout <<"x: " << x << endl;
			cout << "pl: " << p1 << endl;
			cout << "*pl: " << *p1 << endl;
			cout << "animal: " << animal << endl;
			cout << "(int *) animal" << (int*) animal << endl;	
			cout << "p: " << p << endl;
			cout << "(int *) p" << (int*) p << endl;
			cout << "*p: " << *p << endl;
			cout << "bird: " << bird << endl;
			cout << "*bird: " << *bird << endl;
			return 0;
		}
	\end{lstlisting}
	根据输出我们可以看到指针对于字符串和数组的处理方式是不一样的,对于字符串,指针指向这个字符串,那么指向的是这个字符串的首地址,但是直接输出指针的话,输出的是字符串的内容,如果我们对这个指针取"*",即解除引用,输出的就是字符串的第一个字符,如何能得到字符串首字母的地址呢?使用(int *) pointer或者(int *) name\_str就可以了。而对于数组,指针指向的是数组的首地址,那么输出指针,输出的就是这个首地址。程序的输出如下:
	\begin{lstlisting}[language=C++, numbers=left, 
         numberstyle=\tiny,keywordstyle=\color{blue!70},
         commentstyle=\color{red!50!green!50!blue!50},frame=shadowbox,
         rulesepcolor=\color{red!20!green!20!blue!20},basicstyle=\ttfamily]
		x: 0x7ffca993eb00
		pl: 0x7ffca993eb00
		*pl: 1
		animal: bear
		(int*) animal: 0x7ffca993eae0
		p: bear
		(int*) p: 0x7ffca993eae0
		*p: b
		bird: wren
		*bird: w
	\end{lstlisting}
	\item 混合类型指针
\begin{lstlisting}[language=C++, numbers=left, 
         numberstyle=\tiny,keywordstyle=\color{blue!70},
         commentstyle=\color{red!50!green!50!blue!50},frame=shadowbox,
         rulesepcolor=\color{red!20!green!20!blue!20},basicstyle=\ttfamily]
#include <iostream>
#include <cstring>
using namespace std;
struct antarctica_years_end
{
	int year;
};
int main()
{
	antarctica_years_end s1, s2, s3;
	s1.year = 1998;
	antarctica_years_end* pa = &s2;
	pa->year = 1999;
	antarctica_years_end trio[3];
	trio[0].year = 2003;
	cout << trio->year << endl;
	const antarctica_years_end* arp[3] = { &s1, &s2, &s3 };
	cout << arp[0]->year << endl;
	const antarctica_years_end** ppa = arp;
	cout << (*ppa)[0].year << endl;
	cout << (*ppa)->year << endl;
	cout << (*(ppa + 1))->year << endl;
	return 0;
}
\end{lstlisting}	
\end{enumerate}
% section c_指针 (end)

\section{swig对C++库进行python包装}

\subsection{百度DeepSpeech中的swig}


swig版本要求是 >= 3.0, 下载地址为\href{https://sourceforge.net/projects/swig/files/swig/swig-4.0.1/swig-4.0.1.tar.gz/download?use_mirror=nchc}{swig4.0}。

需要boost库,boost下载地址在\href{https://dl.bintray.com/boostorg/release/1.67.0/source/boost_1_67_0.tar.gz}{这里}。boost安装看\href{https://blog.csdn.net/zhangzq86/article/details/81082810}{这里}。



\section{QT with C++}
\subsection{QT中的多线程 Thread}

先简单聊下QT里面的信号与槽,只要是继承自QT的实例,即在定义类之中有 \textcolor{red}{Q_OBJECT} 那么就可以自己去定义信号和槽函数。

具体来说信号的话,我们要想一下什么时候用到信号。除了从 UI 界面直接转到槽,我们还会用到其他信号来触发操作,从代码层面上来说,绑定信号与槽的代码如下:
\begin{lstlisting}[language=C++, numbers=left, 
         numberstyle=\tiny,keywordstyle=\color{blue!70},
         commentstyle=\color{red!50!green!50!blue!50},frame=shadowbox,
         rulesepcolor=\color{red!20!green!20!blue!20},basicstyle=\ttfamily]
connect(sender,SIGNAL(signal), reciver, SLOT(slot_function))
\end{lstlisting}	

简单来说,首先我们需要有一个类,这个类继承自QT,这个类触发了一个信号,通过 emit(signal) 函数表示我发送了这个信号,signal中的参数会传递给后面的 receiver 的槽函数。

使用 wait() 函数会将这个线程跑完之后,再进行后续的一个操作;