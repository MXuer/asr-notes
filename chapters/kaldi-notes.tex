\chapter{Kaldi学习笔记}
\section{kaldi中的数据扰动}
kaldi程序中对原始数据进行扰动以达到数据增强的效果,一般是在单音素对齐,三音素对齐之后,在生成ivector的时候进行扰动处理,扰动有两种方式:速度扰动和音量扰动。如果存在segment文件的话,那么对应的起始和终止时间点会存在segment文件中,提取特征时会根据这个segment中存储的时间节点进行操作;如果不存在的话,相当于整段wav音频都是有效的,那么起始时间点和wav文件相同。扰动也会根据segment文件的有无进行对应的操作。

\subsection{速度扰动}

速度扰动一般是对音频进行加速和减速,根据Povey大佬的论文\href{https://www.danielpovey.com/files/2015_interspeech_augmentation.pdf}{《Audio Augmentation for Speech Recognition》}中的第二部分Audio Perturbation,对mel频谱进行一个偏移就能得到类似加速和减速的效果。首先定义一个扰动因子$\alpha$,假定segment中某一段音频的起始时间和终止时间为$t_1$和$t_2$,那么新的音频起始时间和终止时间计算方式如公式\ref{eqn:sp}。
\begin{align}
\label{eqn:sp}
\begin{split}
  t_{1}' = \frac{t_1}{\alpha} \\
  t_{2}' = \frac{t_2}{\alpha}
\end{split}
\end{align}

kaldi一般取$\alpha$为$0.9$和$1.1$以达到加速和减速的目的。得到了segment文件之后,在wav.scp文件中存储原始音频的位置,加速后音频sox指令和减速后音频sox指令。其详细的脚本指令见代码 utils/data/perturb\_data\_dir\_speed.sh 第74行。最终重新提取特征存于代码根目录下mfcc\_pertubed文件夹中。由于速度扰动对音频时间轴有改动,因此此时需要对音频进行重新对齐的操作。

\subsection{音量扰动}
音量扰动一般是对音频进行增大音量和减小音量。音量增加或者减小的幅度默认取$[0.125, 2]$之间的正态分布值。使用sox工具中的"sox - -vol #volume"来进行实际操作。其详细的脚本指令见代码 utils/data/perturb\_data\_dir\_volume.sh 第71行,其sox操作见代码 utils/data/internal/perturb\_volume.py。其重新提取的特征位于代码根目录下mfcc\_hires 文件夹中。此时由于仅仅对音频进行音量大小的扰动,并没有对时间维度进行操作,因此无需再进行一遍对齐操作,其标签对齐直接采用上一步即速度扰动后生成的对齐结果。此时重新提取特征时,MFCC特征的维度是40d。其原因是40d的MFCC和40d的Fbank维度相同,保存的信息量相似,同时MFCC由于其相关性较弱(DCT去相关),所以能更好的压缩特征,因此Kaldi一般都是采用40d的MFCC作为神经网络的输入特征(见kaldi的各个egs里conf下mfcc\_hires.conf)。
\begin{quotation}
"Config for high-resolution MFCC features, intended for neural network training. Note: we keep all cepstra, so it has the same info as filterbank features, but MFCC is more easily compressible (because less correlated) which is why we prefer this method. "
\end{quotation}

\section{kaldi中的UBM}

\href{http://citeseerx.ist.psu.edu/viewdoc/download?doi=10.1.1.117.338&rep=rep1&type=pdf}{通用背景模型UBM}(Universal Background Model)

\section{kaldi通过lattice输出语音对齐音素和词}

我们通过解码之后得到一堆的 lat.*.gz 文件,这些文件是lattice对齐后生成的对齐文件,其为压缩格式,所以首先通过 gunzip 指令对齐解压,我们以 lat.1.gz 为例来讲这一节的知识点。
\chapter{C++学习笔记}
\begin{lstlisting}[language=shell, numbers=left, 
         numberstyle=\tiny,keywordstyle=\color{blue!70},
         commentstyle=\color{red!50!green!50!blue!50},frame=shadowbox,
         rulesepcolor=\color{red!20!green!20!blue!20},basicstyle=\ttfamily]
gunzip exp/chain/tdnn7q_sp_online/decode_data_tgsmall/lat.1.gz
\end{lstlisting}

这样会在 exp/chain/tdnn7q_sp_online/decode_data_tgsmall/ 下生成一个 lat.1 文件,原先的 lat.1.gz 消失不见了…… lat.1 文件是二进制的格式,其由指令 online2-wav-nnet3-latgen-fatser 生成。