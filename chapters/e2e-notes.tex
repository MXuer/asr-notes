\chapter{端到端语音识别汇总}
%------------------------------------------------------------------------------
%                                CTC 
%------------------------------------------------------------------------------
\section{CTC}
CTC说实话,就是比较麻烦……我已经看过很多遍了,不过看的原理居多,代码这块,前后向的实现以及梯度的求取这些都没看过……先总结下CTC的基本原理吧还是。

\subsection{CTC的原理}

\subsection{CTC的解码}
CTC的解码常用的有两种方式,greedy search和prefix beam search。greedy解码速度很快,但是很容易出错,但是prefix beam search解码速度慢,准确率较高。接下来挨个介绍两种解码方式的算法和流程,以及对应的代码解释。

\subsubsection{greedy search}


\subsubsection{prefix beam search}
First-Pass Large Vocabulary Continuous Speech Recognition using Bi-Directional Recurrent DNNs\upcite{prefix-bs}中针对CTC提出了一种prefix beam search的算法,这种算法能够避免全局搜索的复杂度过高无法实现的问题,同时比greedy search的结果要好很多。

首先介绍下一些公式。公式\ref{eqn:with-lm}是最终的解码公式,配合语言模型和网络输出(声学模型),得到最有可能的$W$词序列。
\begin{align}
\label{eqn:with-lm}
W = \arg\mathop{\max}_{W}p_{net}(W;X)p_{lm}^{\alpha}(W)|W|^{\beta}
\end{align}
其中$p_{net}$是网络的输出,$p_{lm}$是语言模型的输出,$\alpha$和$\beta$分别为语言模型的权重和补偿系数。一般情况下,我们会降低语言模型对整体输出的影响,所以$\alpha$一般取$0.2$ \~ $0.7$。


接下来我们讲一下prefix beam search的demo\upcite{prefix-beam-search}。

总体是有三个循环,第一个是时间维度上的,时间$t$从$1$到$T$,也就是逐帧解码。第二个是对应候选输出序列的,这个是beam search,那么一定得设置一个beam size,那么会考虑所有的候选序列跟当前输出结合起来的概率值,那当前输出的话被剪枝之后还有很多个标签,再挨个的把每一个候选和每一个当前帧的标签进行结合计算。然后再利用这个结合的概率值进行重新排序得到新的候选。

\begin{lstlisting}[language = shell, numbers=left, 
         numberstyle=\tiny,keywordstyle=\color{blue!70},
         commentstyle=\color{red!50!green!50!blue!50},frame=shadowbox,
         rulesepcolor=\color{red!20!green!20!blue!20},basicstyle=\ttfamily]
pruned_alphabet = [alphabet[i] for i in np.where(ctc[t] > prune)[0]]
\end{lstlisting}

这一步是为了减少计算,也就是说先设定一个阈值,当前$t$时刻的时候,会做一个判断,只有当前时刻各个标签概率值大于 prune 的时候才会去做后续的操作,这就意味着当前时刻所有概率低于 prune 的标签都会被抛弃,不再参与到当前时刻的解码过程中。因为这些标签概率值太小,不太可能是正确的输出标签,留着只会增加计算量,还不如删掉,省时省心省力!

\begin{lstlisting}[language = shell, numbers=left, 
         numberstyle=\tiny,keywordstyle=\color{blue!70},
         commentstyle=\color{red!50!green!50!blue!50},frame=shadowbox,
         rulesepcolor=\color{red!20!green!20!blue!20},basicstyle=\ttfamily]
if len(l) > 0 and l[-1] == '>':
  Pb[t][l] = Pb[t - 1][l]
  Pnb[t][l] = Pnb[t - 1][l]
  continue 
\end{lstlisting}

这一步就是判断下是不是到结尾了,结尾的表示符号是">"。如果到了结尾呢,说明这个时候输出的标签序列和$t-1$时刻是一毛一样滴。$t$时刻输出序列$l$以 blank 结尾的概率跟$t-1$时刻以 blank 结尾的概率是一样的,$t$时刻输出序列$l$以 non-blank 结尾的概率跟$t-1$时刻以 non-blank 结尾的概率是一样的。然后就跳出当前时刻,因为当前时刻表示这个序列已经到了结尾了,没必要再折腾下去了。
 
\begin{lstlisting}[language = shell, numbers=left, 
         numberstyle=\tiny,keywordstyle=\color{blue!70},
         commentstyle=\color{red!50!green!50!blue!50},frame=shadowbox,
         rulesepcolor=\color{red!20!green!20!blue!20},basicstyle=\ttfamily]
if c == '%':
  Pb[t][l] += ctc[t][-1] * (Pb[t - 1][l] + Pnb[t - 1][l])
\end{lstlisting}

我们假设$\%$代表blank这个标签。剪枝之后,会对还剩下的那些标签走一遍遍历,每一个标签都会尝试着和之前存起来的候选序列进行结合,算出来一个概率值。那么既然是遍历,当然会轮到牛逼的 blank。所以首先看看当前的这个标签是不是blank。如果是blank的话,我们就没必要对当前的这个候选序列做啥子改动了,也就是当前时刻的输出标签序列还是$l$,因为最终输出的时候,blank也不会出现。既然当前这个标签是blank,那么$t$时刻的标签序列$l$的概率需要和当前时刻输出为 blank 的概率结合一下,变成当前时刻的 $Pb[t][l]$。其计算公式从代码里就可以看到。

\begin{lstlisting}[language = shell, numbers=left, 
         numberstyle=\tiny,keywordstyle=\color{blue!70},
         commentstyle=\color{red!50!green!50!blue!50},frame=shadowbox,
         rulesepcolor=\color{red!20!green!20!blue!20},basicstyle=\ttfamily]
l_plus = l + c
if len(l) > 0 and c == l[-1]:
  Pnb[t][l_plus] += ctc[t][c_ix] * Pb[t - 1][l]
  Pnb[t][l] += ctc[t][c_ix] * Pnb[t - 1][l]
\end{lstlisting}

如果说当前时刻的标签不是 blank,而是上个时刻的这个候选序列的最后一个输出标签,也就是说当前时刻的输出和上一个时刻的候选序列尾部产生了重复,这个时候其实是有两种情况的,第一种情况是上一个时刻的输出标签正好是 blank,因为从上面那一步代码中我们可以看出来,候选序列中是不会出现blank的,那如果是这种的情况,说明实际的输出序列就是有两个一样的字母,输出就是$l\_plus$,其尾部有两个一样的字母,这个时候候选序列的概率就等于当前时刻的标签概率乘以上一个时刻输出为blank的序列概率,也就是$Pb[t - 1][l]$;第二种情况是上一个时刻的输出确实也是这个标签,那么说明这个时候的候选序列不需要做啥变动,但是概率值需要变一下,当前时刻标签概率乘以上一个时刻输出是 non-blank 的序列概率值。

\begin{figure}[h]
  \centering
  \includegraphics[width=0.6\textwidth]{error-prefix}
  \caption{Prefix Beam Search原论文中算法错误地方 \label{fig:error-prefix}}
\end{figure}

另外原论文中关于这一块的计算步骤写错了,也就是算法中的这一步,如图\ref{fig:error-prefix}。简直坑死个人。

\begin{lstlisting}[language = shell, numbers=left, 
         numberstyle=\tiny,keywordstyle=\color{blue!70},
         commentstyle=\color{red!50!green!50!blue!50},frame=shadowbox,
         rulesepcolor=\color{red!20!green!20!blue!20},basicstyle=\ttfamily]
elif len(l.replace(' ', '')) > 0 and c in (' ', '>'):
  lm_prob = lm(l_plus.strip(' >')) ** alpha
  Pnb[t][l_plus] += lm_prob * ctc[t][c_ix] * (Pb[t - 1][l] + Pnb[t - 1][l])
\end{lstlisting}

那还有可能当前的输出是 ' '(space),也就是说输出是空格或者是结尾,这个时候说明一个完整的词出现了,我们就可以利用语言模型(LM)来对输出进行修正和约束,避免出现毫无意义的结果。那么这个词代入到语言模型中会出现一个概率值,当前候选序列的概率值就通过公式\ref{eqn:with-lm}来计算,从代码中也可以看出来。

\begin{lstlisting}[language = shell, numbers=left, 
         numberstyle=\tiny,keywordstyle=\color{blue!70},
         commentstyle=\color{red!50!green!50!blue!50},frame=shadowbox,
         rulesepcolor=\color{red!20!green!20!blue!20},basicstyle=\ttfamily]
Pnb[t][l_plus] += ctc[t][c_ix] * (Pb[t - 1][l] + Pnb[t - 1][l])
\end{lstlisting}

还有最后一种情况,就是既不是 blank,又不是 space,当前输出标签和候选标签序列的最后一个字母也不一样,这个时候,就直接算出候选标签序列和当前标签的概率乘积就行,候选标签序列也有两种情况:上一个时刻以blank或者以non-blank结尾。综上集中基本的情况都已经讲完了。

\begin{lstlisting}[language = shell, numbers=left, 
         numberstyle=\tiny,keywordstyle=\color{blue!70},
         commentstyle=\color{red!50!green!50!blue!50},frame=shadowbox,
         rulesepcolor=\color{red!20!green!20!blue!20},basicstyle=\ttfamily]
if l_plus not in A_prev:
  Pb[t][l_plus] += ctc[t][-1] * (Pb[t - 1][l_plus] + Pnb[t - 1][l_plus])
  Pnb[t][l_plus] += ctc[t][c_ix] * Pnb[t - 1][l_plus]
\end{lstlisting}

按照原始论文中,还有上面这个公式。也就是说算出来的$l\_plus$不在候选标签序列里面,就会去之前时刻的候选序列里面去找,再利用之前的后续序列概率计算当前的概率值。百度的Deep Speech2代码里面说:这个部分不知道在干啥,还没啥用,所以就给去掉了。

我觉得……也不太好理解……

讲到这儿核心的代码部分已经讲完了,剩下的就是把当前时刻的标签,不管是以blank结尾的还是非blank结尾的综合起来,然后进行排序,根据beam size的大小得到新的候选序列,如此循环往复,直到这个序列输出到了尽头,就可以得到最终的结果啦\~\~\~

完整代码如下:

%%%%%%%%%%%%%%%%%%%%%%%Code for Prefix Beam Search%%%%%%%%%%%%%%%%%%%%%%%%%%%%%%%%%
\begin{lstlisting}[language = python, numbers=left, 
         numberstyle=\tiny,keywordstyle=\color{blue!70},
         commentstyle=\color{red!50!green!50!blue!50},frame=shadowbox,
         rulesepcolor=\color{red!20!green!20!blue!20},basicstyle=\ttfamily]
from collections import defaultdict, Counter
from string import ascii_lowercase
import re
import numpy as np

def prefix_beam_search(ctc, lm=None, k=25, alpha=0.30, beta=5, prune=0.001):
  """
  Performs prefix beam search on the output of a CTC network.

  Args:
    ctc (np.ndarray): The CTC output. Should be a 2D array (timesteps x alphabet_size)
    lm (func): Language model function. Should take as input a string and output a probability.
    k (int): The beam width. Will keep the 'k' most likely candidates at each timestep.
    alpha (float): The language model weight. Should usually be between 0 and 1.
    beta (float): The language model compensation term. The higher the 'alpha', the higher the 'beta'.
    prune (float): Only extend prefixes with chars with an emission probability higher than 'prune'.

  Retruns:
    string: The decoded CTC output.
  """

  lm = (lambda l: 1) if lm is None else lm # if no LM is provided, just set to function returning 1
  W = lambda l: re.findall(r'\w+[\s|>]', l)
  alphabet = list(ascii_lowercase) + [' ', '>', '%']
  F = ctc.shape[1]
  ctc = np.vstack((np.zeros(F), ctc)) # just add an imaginative zero'th step (will make indexing more intuitive)
  T = ctc.shape[0]

  # STEP 1: Initiliazation
  O = ''
  Pb, Pnb = defaultdict(Counter), defaultdict(Counter)
  Pb[0][O] = 1
  Pnb[0][O] = 0
  A_prev = [O]
  # END: STEP 1

  # STEP 2: Iterations and pruning
  for t in range(1, T):
    pruned_alphabet = [alphabet[i] for i in np.where(ctc[t] > prune)[0]]
    for l in A_prev:
      
      if len(l) > 0 and l[-1] == '>':
        Pb[t][l] = Pb[t - 1][l]
        Pnb[t][l] = Pnb[t - 1][l]
        continue  

      for c in pruned_alphabet:
        c_ix = alphabet.index(c)
        # END: STEP 2
        
        # STEP 3: “Extending” with a blank
        if c == '%':
          Pb[t][l] += ctc[t][-1] * (Pb[t - 1][l] + Pnb[t - 1][l])
        # END: STEP 3
        
        # STEP 4: Extending with the end character
        else:
          l_plus = l + c
          if len(l) > 0 and c == l[-1]:
            Pnb[t][l_plus] += ctc[t][c_ix] * Pb[t - 1][l]
            Pnb[t][l] += ctc[t][c_ix] * Pnb[t - 1][l]
        # END: STEP 4

          # STEP 5: Extending with any other non-blank character and LM constraints
          elif len(l.replace(' ', '')) > 0 and c in (' ', '>'):
            lm_prob = lm(l_plus.strip(' >')) ** alpha
            Pnb[t][l_plus] += lm_prob * ctc[t][c_ix] * (Pb[t - 1][l] + Pnb[t - 1][l])
          else:
            Pnb[t][l_plus] += ctc[t][c_ix] * (Pb[t - 1][l] + Pnb[t - 1][l])
          # END: STEP 5

          # STEP 6: Make use of discarded prefixes
          if l_plus not in A_prev:
            Pb[t][l_plus] += ctc[t][-1] * (Pb[t - 1][l_plus] + Pnb[t - 1][l_plus])
            Pnb[t][l_plus] += ctc[t][c_ix] * Pnb[t - 1][l_plus]
          # END: STEP 6

    # STEP 7: Select most probable prefixes
    A_next = Pb[t] + Pnb[t]
    sorter = lambda l: A_next[l] * (len(W(l)) + 1) ** beta
    A_prev = sorted(A_next, key=sorter, reverse=True)[:k]
    # END: STEP 7

  return A_prev[0].strip('>')

\end{lstlisting}

%------------------------------------------------------------------------------
%                                RNN Tranducer 
%------------------------------------------------------------------------------
\section{RNN-Tranducer}

%------------------------------------------------------------------------------
%                                Attention 
%------------------------------------------------------------------------------
\section{Attention}

%------------------------------------------------------------------------------
%                                Transformer 
%------------------------------------------------------------------------------
\section{Transformer}


%------------------------------------------------------------------------------
%                                CNNs 
%------------------------------------------------------------------------------
\section{CNNs}

%------------------------------------------------------------------------------
%                                Mixed Models 
%------------------------------------------------------------------------------
\section{Mixed Models}

\subsection{Self-Attention Transducers for End-to-End Speech Recognition}
这篇论文的作者是田正坤,来自中国科学院自动化所。本论文的主要贡献有:
\begin{enumerate}
  \item 用self-attention模块替代了原来RNN-T中的RNN部分,可以用于并行计算;
  \item 利用 path-aware regularization 帮助SA-T学习对齐;
  \item 使用了chunk-flow机制来进行解码。
\end{enumerate}

\subsubsection{SA-T的基本结构}

\subsubsection{path-aware regularization}

\subsubsection{chunk flow mechanism}

