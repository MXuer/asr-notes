\chapter{Docker}

\section{常用操作}
镜像(image)和容器(container)的关系,就像是面向对象程序设计中的类和实例一样,镜像是静态的定义,容器是镜像运行时的实体。容器可以被创建、启动、停止、删除和暂停等。

容器的实质是进程,但与直接在宿主机执行的进行不同,容器进程运行于属于自己的独立的命名空间。
\begin{lstlisting}[language = shell, numbers=left, 
         numberstyle=\tiny,keywordstyle=\color{blue!70},
         commentstyle=\color{red!50!green!50!blue!50},frame=shadowbox,
         rulesepcolor=\color{red!20!green!20!blue!20},basicstyle=\ttfamily]
#开启镜像
sudo nvidia-docker run -it -v $(pwd)/DeepSpeech:/DeepSpeech -v /data1/asr_data:/mnt/data -v //data/kaldi/2019_0521_kaldi/kaldi-master:/mnt/kaldi paddlepaddle/deep_speech:latest-gpu /bin/bash
# 挂起镜像
Ctrl+P+Q
#运行已挂起的镜像
docker attach $CONTAINER_ID
\end{lstlisting}

\section{实践要求}
docker如果想要挂载上本地的IP地址,可以再运行docker的时候加上指令 --net=host。如果要挂载本地物理磁盘,加上指令 -v。如果要将宿主机的端口与容器的端口绑定,可以使用 -p <宿主机端口>:<容器端口>。
\begin{lstlisting}[language = shell, numbers=left, 
         numberstyle=\tiny,keywordstyle=\color{blue!70},
         commentstyle=\color{red!50!green!50!blue!50},frame=shadowbox,
         rulesepcolor=\color{red!20!green!20!blue!20},basicstyle=\ttfamily]
nvidia-docker run -it --net=host  -p 50001:22 -v $(pwd)/DeepSpeech:/DeepSpeech  -v /data1/asr_data:/mnt/data -v /data/kaldi/2019_0521_kaldi/kaldi-master:/mnt/kaldi duhu/ds-server /bin/bash
\end{lstlisting}

\begin{enumerate}
  \item 容器不应该向其存储层内写入任何数据,容器存储层要保持无状态化。所有的文件写入操作,都应该使用数据卷(Volume)、或者绑定宿主目录,在这些位置的读写会跳过容器存储层,直接对宿主(或网络存储)发生读写,其性能和稳定性更高。
  \item 数据卷的生存周期独立于容器,容器消亡,数据卷不会消亡。因此,使用数据卷后,容器删除或者重新运行之后,数据却不会丢失。
\end{enumerate}

\section{docker安装TensorFlow}
为了避免影响到主机上诸多配置,因此选用docker安装TensorFlow,想怎么造就怎么造。安装TensorFlow时,使用以下指令就会启动该镜像安装。
\begin{lstlisting}[language = shell, numbers=left, 
         numberstyle=\tiny,keywordstyle=\color{blue!70},
         commentstyle=\color{red!50!green!50!blue!50},frame=shadowbox,
         rulesepcolor=\color{red!20!green!20!blue!20},basicstyle=\ttfamily]
docker pull  tensorflow/tensorflow
\end{lstlisting}