\chapter{Windows相关}
本章用于记录一些Windows使用的一些问题,方便再遇到的时候查询。
\section{win10 .net framework 3.5 安装报错 0x800F0954问题}
本问题解决参考\href{https://blog.csdn.net/asd77882566/article/details/80024043}{tOneDay}的博客,其他的一些博客都没有解决问题。以此为准:
\begin{enumerate}
	\item 打开注册表:cmd+r 输入regedit,确定;
	\item 找到路径HKEY\_LOCAL\_MACHINE-SOFTWARE-Policies-Microsoft-Windows-WindowsUpdate-AU,其中UseWUServer默认值为1,改成0;
	\item 打开服务列表,重启Windows Update service;
	\item 此时可以正常安装.net framework 3.5;
	\item 将第二步的修改还原,并重启Windows Update service。
\end{enumerate}

\section{Win10 安装虚拟机}
\href{https://blog.csdn.net/weixin_44779019/article/details/95219733?utm_source=distribute.pc_relevant.none-task}{win10系统1903版 VMware Workstation 与 Device/Credential Guard 不兼容.在禁用 Device/Credenti}