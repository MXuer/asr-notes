\chapter{深度学习框架笔记}
\section{paddlepaddle}
本处记录一些用paddlepaddle跑百度的DeepSpeech2的笔记和问题:
\begin{enumerate}
	\item 基于一些原因,服务器A上的GPU暂时不可用,因此在服务器B上重新部署了百度基于paddlepaddle的\href{https://github.com/PaddlePaddle/DeepSpeech}{DS2}。A和B有共享目录,因此这个repository是放在一个共享目录下的,但是在运行./run\_train.sh之后,模型初始化没有任何问题,log显示已经开始在训练了,但是在存模型那块死活不动。因为之前在服务器A上训练过一段时间,所以对应存储checkpoint的路径是服务器A的用户创建的,如果没有给够权限的话,服务器B是没有办法存储checkpoint到原来服务器A的那个文件夹下的,所以两种办法:给够路径足够的权限或者以服务器B的用户重新创建一个路径;
	\item 在利用训练和测试数据生成vocab.txt的时候,因为是从对应的文本中提取的vocab.txt,不同的数据库文本不一样,所以最终生成的vocab.txt也是不一样的。千万不要用其他数据库的vocab.txt来对当前库进行训练或者测试,因为结果会惨不忍睹;
	\item DS2提取的音频特征是80维的fbank。
\end{enumerate}}

\section{pytorch}

\section{tensorflow}
打今儿起,我要开始学习TensorFlow 2.0了。所以事无巨细,笔记在此。


\subsection{TensorFlow中的各类操作}

函数求导:

变量与普通张量的一个重要区别是其默认能够被TensorFlow的自动求导机制所求导,因此往往被定义为机器学习模型的参数。以下分别为一元函数$y=x^{2}$求导和多元函数$L(w, b)=||Xw+b-y||^{2}$求导。
\begin{lstlisting}[language=python, 
         numberstyle=\tiny,keywordstyle=\color{blue!70},
         commentstyle=\color{red!50!green!50!blue!50},frame=shadowbox,
         rulesepcolor=\color{red!20!green!20!blue!20},basicstyle=\ttfamily]
import tensorflow as tf
x = tf.Variable(initial_value=3.)
with tf.GradientTape() as tape:
	y = tf.square(x)
y_gradient = tape.gradient(y, x)
print([y, y_gradient])
\end{lstlisting}
\begin{lstlisting}[language=python, 
         numberstyle=\tiny,keywordstyle=\color{blue!70},
         commentstyle=\color{red!50!green!50!blue!50},frame=shadowbox,
         rulesepcolor=\color{red!20!green!20!blue!20},basicstyle=\ttfamily]
import tensorflow as tf
X = tf.constant([[1.,2.], [3.,4.]])
y = tf.constant([[1.],[2.]])
w = tf.Variable(initial_value=[[1.],[2.]])
b = tf.Variable(initial_value=1.)
with tf.GradientTape() as tape:
    L = 0.5*tf.reduce_sum(tf.square(tf.matmul(X, w)+b-y))
w_grad, b_grad = tape.gradient(L,[w,b])
print(test.numpy())
print(L.numpy())
print(w_grad.numpy())
print(b_grad.numpy())
\end{lstlisting}

利用线性回归的例子来总结下TensorFlow的训练步骤:
\begin{enumerate}
	\item 准备数据,数据归一化,将训练数据设定为\textcolor{gray}{x = tf.constant(x)};
	\item 设定变量和初始化变量,\textcolor{gray}{w = tf.Variable(initial\_value=tf.random.uniform(shape=shape};
	\item 设定超参数,\textcolor{gray}{learningrate, num\_epochs};
	\item 选择优化器,\textcolor{gray}{optimizer = tf.keras.optimizers.SGD(learning\_rate=learning\_rate);
	\item 循环迭代:
		\begin{itemize}
			\item 使用TensorFlow的自动求导机制记录损失函数的梯度信息,\textcolor{gray}{with tf.GradientTape as tape:}
			\item 使用 \textcolor{gray}{tape.gradient(y, variables)}
			\item 使用 \textcolor{gray}{optimizers.apply\_gradients(grads_and\_vars=zip(grads, variables))}
		\end{itemize}
\end{enumerate}


\subsection{如何利用TensorFlow快速搭建动态模型}
\begin{itemize}
	\item 模型的构建:\textcolor{red}{tf.keras.Model}和\textcolor{red}{tf.keras.layers}
	\item 模型的损失函数:\textcolor{red}{tf.keras.losses}
	\item 模型的优化器:\textcolor{red}{tf.keras.optimizer}
	\item 模型的评估:\textcolor{red}{tf.keras.metrics}
\end{itemize}

